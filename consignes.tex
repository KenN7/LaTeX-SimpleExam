\section*{Consignes :}
\begin{itemize}
\item Complétez dès maintenant l'encadré en bas de cette page (nom, prénom) – \textbf{N'indiquez \textbf{PAS}} votre nom sur les autres feuilles (pour permettre la correction en aveugle).\\
\item Cette partie de l'examen dure \textbf{2H}.
\item Documents et calculatrice \textbf{SONT AUTORISÉS}. %à l'exception des énoncés et des corrigés des années précédentes.
\item Répondez directement dans les cases prévues à cet effet : la longueur de l'emplacement vous indique à peu près la longueur du développement attendu.
\item Efforcez-vous d'écrire le plus clairement et le plus lisiblement possible.\\
\item Indiquez les \textbf{unités} de toutes les grandeurs chiffrées intervenant dans les réponses aux questions.
\item Utilisez les \textbf{approximations} légitimes permettant de gagner du temps.
%\item Lors de la résolution éventuelle d'un schéma, précisez très clairement les \textbf{notations} et les \textbf{conventions} utilisées.
\item Sauf indication contraire, décrivez le \textbf{raisonnement} qui a conduit à chaque réponse (\textbf{la longueur de l'emplacement prévu pour la réponse vous indique approximativement la longueur du développement attendu}).\\
\item Cet examen comporte \total{Qcount} questions pour un total de \ref{totpoints}~points %\total{TotalPoints}~points.
\item Vérifiez que votre exemplaire contient bien \pageref{LastPage} pages numérotées de 1 à \pageref{LastPage}.
\item Vous trouverez \total{totdraft}~feuille(s) de brouillon à la fin de cet examen, vous \textbf{pouvez} détacher ces feuilles.\textbf{Aucune} réponse écrite sur un brouillon ne pourra être prise en compte dans la correction.
\end{itemize}
