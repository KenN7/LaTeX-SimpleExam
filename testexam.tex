\documentclass{exam}
\usepackage[utf8x]{inputenc}

\usepackage[frenchb]{babel}
\usepackage[T1]{fontenc}

\usepackage{graphicx}
\usepackage{amssymb}
\usepackage{amsmath}
\usepackage{wasysym} %smiley
\usepackage{hyperref}% hyperliens
\usepackage{tikz}
\usetikzlibrary{babel,positioning,calc}
\usepackage[]{circuitikz}

\usepackage{pdfpages}

\langexam{frenchb}
\correction{false}
\toptitle{Électronique numérique}{1ière session}
\author{Ken Hasselmann}

\begin{document}

\changedate{06}{10}{2016}
\examtitle{Examen d'électronique numérique}{Seconde partie: \og cahier ouvert\fg}

\frontpage{consignes.tex}
\artRGE{true}
\namebox{name}
\newpage

\section{Faire un four...}

Il vous est demandé de concevoir un four commandé par un dsPIC33 alimenté en 3.3~V identique à celui utilisé durant les laboratoires, avec un cycle d'instruction à 40~MHz.

\begin{itemize}
	\item Le tableau de commande comprend~:
	\begin{itemize}
		\item Un afficheur LCD avec une interface serie RS-232.
		\item Un sélecteur de programme et un sélecteur de température
	\end{itemize}

	\item Il est possible de sélectionner 5 niveaux de température différents.
	Lorsque le four est allumé, l'afficheur affiche la température actuelle du four.
	Lorsque le four est éteint, l'afficheur affiche l'heure (on ne s'occupera pas du réglage de l'heure).

	\item Une thermistance dépendant linéairement de la température du four est parcourue par un courant de 1~mA et fournit une tension entre 10~mV et 100~mV (correspondant respectivement à $25$ et $250$).
	Cette température sera échantillonnée par le dsPIC33 à une fréquence de 1~Hz.

	\item La puissance du four est réglée grâce à un signal PWM envoyé par le dsPIC33.
	La puissance maximale du four est de 3000~W.
	La période du PWM est de 100~ms.

\end{itemize}

\Question{20}{3}
{
	Représentez le système sous la forme d'un schéma-bloc détaillé contenant le C, les différents périphériques du problème et les éventuels circuits d'interfaçage.

}
{}

\Question{35}{1.5}
{
    Combien de lettres y a-t-il dans le mot mot ?
	\addgrid{7}{5}{axes}
}
{}

\adddraft{1}{}

\end{document}
